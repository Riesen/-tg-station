\documentclass[10pt, a4paper]{article}

\usepackage[utf8]{inputenc}
\usepackage{amsmath}
\usepackage{amssymb}

\title{NanoUI Template Reference}
\author{Zaers}
\begin{document}


\begin{titlepage}
 \maketitle
 \tableofcontents
\end{titlepage}
\section{Syntax}

The basic syntax of the templates could be described as HTML with JSrender extensions for displaying dynamic data.

The CSS classes and icons used in NanoUI are defined in /nano/css/*.css

JSrender functionality is called between \{\{ \}\}.

The variables used are defined in the game code for each particular object that use nanoUI

\subsection{Displaying a variable}
Variables are displayed via \{\{:\verb|variable|\}\}

\subsection{Calling a function}
Functions are called using \{\{\textasciitilde\verb|function(ARGS)|\}\}

\subsection{Conditionals}
JSrender conditionals are used via 

\{\{if \verb|condition|\}\}

and 

\{\{else \verb|optional contidition|\}\}

terminated with

\{\{/if\}\}

\subsection{Progress bars}
Progress bars are used as a function via 

\verb|displayBar(variable, minimum, maximum, styleClass, text)|

\subsection{Links}
BYOND links are generated as a function via

\verb|link(text, icon, parameters, status, elementClass, elementId)|

\subsection{For loops}
JSrender supports for loops, used as

\{\{for \verb|object or array|\}\}

\verb|...|

\{\{/for\}\}

\subsection{Data linking}
JSViews data linking is done via 
\{\textasciicircum\{ \}\}
and automatically refreshes the data displayed if it changes

\section{Other resources}
If the above reference was insufficient, 
JSrender API documentation is available at \verb|http://www.jsviews.com|

and complete templates are viewable in /nano/templates/*.tmpl

\end{document}
